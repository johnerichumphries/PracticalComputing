
%Input preamble
%Style
\documentclass[11pt]{article}
\usepackage[top=1in, bottom=1in, left=1in, right=1in]{geometry}
\parindent 22pt

%Packages
\usepackage{amsmath}
\usepackage{amsfonts}
\usepackage{amssymb}
\usepackage{bm}
\usepackage[table]{xcolor}
\usepackage{tabu}
\usepackage{makecell}
\usepackage{longtable}
\usepackage{multirow}
\usepackage[normalem]{ulem}
\usepackage{etoolbox}
\usepackage{graphicx}
\usepackage{tabularx}
\usepackage{ragged2e} 
\usepackage{booktabs}
\usepackage{caption}
\usepackage[none]{hyphenat}
\usepackage{fixltx2e}
\usepackage{threeparttablex}
\usepackage[capposition=top]{floatrow}
\usepackage{subcaption}
\usepackage{pdfpages}
\usepackage{pdflscape}
\usepackage{natbib}
\definecolor{maroon}{HTML}{990012}
\usepackage[colorlinks=true,linkcolor=maroon,citecolor=maroon]{hyperref}
\usepackage{tikz}
\usetikzlibrary{shapes}
\usepackage{setspace}

%Functions
\DeclareMathOperator{\cov}{Cov}
\DeclareMathOperator{\var}{Var}
\DeclareMathOperator{\plim}{plim}

%Math Environments
\newtheorem{theorem}{Theorem}[section]
\newtheorem{claim}[theorem]{Claim}
\newtheorem{assumption}[theorem]{Assumption}
\newtheorem{definition}[theorem]{Definition}
\newtheorem{hypothesis}[theorem]{Hypothesis}
\newtheorem{property}[theorem]{Property}
\newtheorem{example}[theorem]{Example}
\newtheorem{exercise}[theorem]{Exercise}
\newtheorem{condition}[theorem]{Condition}
\newtheorem{remark}[theorem]{Remark}
\newenvironment{proof}{\paragraph{Proof:}}{\hfill$\square$}

%Commands
\newcommand\independent{\protect\mathpalette{\protect\independenT}{\perp}}
\def\independenT#1#2{\mathrel{\rlap{$#1#2$}\mkern2mu{#1#2}}}
\newcommand{\overbar}[1]{\mkern 1.5mu\overline{\mkern-1.5mu#1\mkern-1.5mu}\mkern 1.5mu}
\newcommand{\equald}{\ensuremath{\overset{d}{=}}}

%Logo
%\AddToShipoutPictureBG{%
 % \AtPageUpperLeft{\raisebox{-\height}{\includegraphics[width=1.5cm]{uchicago.png}}}
%}




\begin{document}

\title{\textbf{Practical Computing for Economists}\footnote{This course does not count as a general distribution requirement and does not assign other grade than ``R''. }}
\author{Spring 2015}
\date{First Draft: July 24 , 2014 \\ This Draft: \today}
\maketitle

\begin{abstract}
\noindent The objective of this course is to provide students with the basic knowledge economists need when performing economic analysis that is too complicated to solve by hand. Various programs will be reviewed, particularly focusing on computational advantages and time allocation trade-offs when solving typical problems economists face (e.g. numerical integration, likelihood maximization, bootstrap standard errors). The course is entirely practical and based on our experience as researchers. This is not a theoretical course; for example, we do not derive convergence properties of numerical optimizers. Instead, the main goal is to provide economists with the essential computational toolkit to conduct high-level research efficiently and accurately.
\end{abstract}


\section{A Rapid Introduction to R for Economists}
\noindent \textbf{Instructor}: John Eric Humphries.\\
\noindent \textbf{Time}: Two weeks.\\
\noindent \textbf{Instruction Material}: Available on \href{http://johnerichumphries.com}{http://johnerichumphries.com}.\\
\noindent \textbf{Description:} this section provides a rapid introduction to the R environment for statistical computing. The objective is to equip students with the following skills: rads and understand R code; write clear, documented, and reusable R code; load and explore real-world data; visualize and summarize results; implement own economietric methods; call an optimizer to maximize a function; Gaussian quadrature to maximize a function (if time allows); basic parallel computing (if time allows). 

\section{An Introduction to Rapid R for Economists}
\noindent \textbf{Instructor}: Philip Barrett.\\
\noindent \textbf{Time}: Two weeks.\\
\noindent \textbf{Instruction Material}: \\
\noindent \textbf{Description:} this section is about showing you how to speed up your R code.  We will very briefly discuss parallelism, but our focus will be on learning how to rewrite loops and linear algebra in C++.  These are often the most computationally intensive parts of a large project and can be pushed to C++ very easily and with substantial speed gains - improvements of 60 to 100 times faster are not uncommon.  We will use the \tt Rcpp \rm package to write C++ code that we can call from directly R, and the \tt RcppArmadillo \rm package to use very fast C++ linear algebra libraries with really easy syntax.

\section{Data Manipulation, Data Management, and Data Storage}
\noindent \textbf{Instructor}: Jorn Boehnke.\\
\noindent \textbf{Time}: Two weeks.\\
\noindent \textbf{Instruction Material}: \\
\noindent \textbf{Description:} this section covers the following. (i) Data structuring: It is very important to know about data types. It is a classification identifying one of various types of data, such as real, integer or boolean, that determines the possible values for that type. Storing numbers as texts, for example, will reduce efficiency and impede value comparison. Every data type has its own effective way to store it.\\
\indent (ii) RegEx: Regular expressions (RegEx) is a very powerful text editing tool and an important part of this workshop. Any modern programming language (e.g. Java, C++, VB), statistical software (e.g. R, Stata, MATLAB) and advanced plain text editor (e.g. Notepad++, TextWrangler, Sublime Text) are capable of RegEx. It provides a concise and flexible means to match (specify and recognize) strings of text, such as particular characters, words, or patterns of characters. With its functionality, RegEx offers a very efficient and easy way to edit / manipulate text‐based data.\\
\indent (iii)MySQL: MySQL is the world’s second most widely used database. It is the most widely used open source database.1 You will most likely receive an SQL dump if you apply for data from an internet company. (My)SQL interfaces well with programming language and statistical software. It allows you to query / alter subsets of the data easy and fast. You can think of (My)SQL as free Microsoft Excel on speed, and without row limitations.\\
\indent (iv )MongoDB \& JSON: MongoDB is the world’s most widely used NonSQL DB. You can store anything in it, and MongoDB will do the magic for you. It is the natural choice if you are having a hard time defining a column structure for your data. JSON stands for JavaScript Object Notation and is the data structuring language MongoDB operates in. It is, however, not limited to MongoDB. The Twitter firehose application, for example, required knowledge of JSON.
\section{Structural Econometrics 1: High-Dimensional Panel Data Estimators and Multiple Hypothesis Testing}
\noindent \textbf{Instructor}: Bradley J. Setzler.\\
\noindent \textbf{Time}: Two weeks.\\
\noindent \textbf{Instruction Material}: Available on \href{JuliaEconomics.com}{JuliaEconomics.com}.\\
\noindent \textbf{Description}: this section demonstrates the estimation of high-dimensional panel data models, including fixed effects and recursive Bayesian filters, through numerical maximum likelihood estimation. Recognizing the multiple testing problem induced by estimating hundreds of parameters simultaneously, we use cluster parallelization to perform non-parametric multiple testing corrections. Always, the aim is to write estimators in computer code that exactly imitate the natural syntax of economics, so that we could copy our \LaTeX\ code directly into a numerical program with minimal effort. Always, we use simulation of the assumed data generating process to verify that our code can retrieve the true parameter values. Instruction materials are available in Julia (primary), Python, and R.


\section{Structural Econometrics 2: Dynamic Programming Estimators}
\noindent \textbf{Instructor}: Jorge L. Garc\'{i}a.\\
\noindent \textbf{Time}: Two weeks.\\
\noindent \textbf{Instruction Material}: \href{http://home.uchicago.edu/~jorgelgarcia/}{\begin{scriptsize}http://home.uchicago.edu/~jorgelgarcia\end{scriptsize}}.
\noindent \textbf{Description}: In this section structural models in Economics will be described briefly. In addition, structural and non-structural estimation will be recognized as approaches that enable the user to answer different research questions. Also, parametric and non-parametric assumptions will be discussed as auxiliary means that contribute to recover parameters. This section of the course will build on \citet{keane2011structural} but will be expanded by providing a wide range of computational implementations of their baseline model – a classic discrete choice dynamic programming model of women's labor supply –. The program Julia will be used for these implementations and additional some Python spinets will be provided for the sake of comparison.


\bibliographystyle{chicago}
\bibliography{BibtexFiles/Syllabus}
\clearpage

\end{document}