
%Input preamble
%Style
\documentclass[11pt]{article}
\usepackage[top=1in, bottom=1in, left=1in, right=1in]{geometry}
\parindent 22pt

%Packages
\usepackage{amsmath}
\usepackage{amsfonts}
\usepackage{amssymb}
\usepackage{bm}
\usepackage[table]{xcolor}
\usepackage{tabu}
\usepackage{makecell}
\usepackage{longtable}
\usepackage{multirow}
\usepackage[normalem]{ulem}
\usepackage{etoolbox}
\usepackage{graphicx}
\usepackage{tabularx}
\usepackage{ragged2e} 
\usepackage{booktabs}
\usepackage{caption}
\usepackage[none]{hyphenat}
\usepackage{fixltx2e}
\usepackage{threeparttablex}
\usepackage[capposition=top]{floatrow}
\usepackage{subcaption}
\usepackage{pdfpages}
\usepackage{pdflscape}
\usepackage{natbib}
\definecolor{maroon}{HTML}{990012}
\usepackage[colorlinks=true,linkcolor=maroon,citecolor=maroon]{hyperref}
\usepackage{tikz}
\usetikzlibrary{shapes}
\usepackage{setspace}

%Functions
\DeclareMathOperator{\cov}{Cov}
\DeclareMathOperator{\var}{Var}
\DeclareMathOperator{\plim}{plim}

%Math Environments
\newtheorem{theorem}{Theorem}[section]
\newtheorem{claim}[theorem]{Claim}
\newtheorem{assumption}[theorem]{Assumption}
\newtheorem{definition}[theorem]{Definition}
\newtheorem{hypothesis}[theorem]{Hypothesis}
\newtheorem{property}[theorem]{Property}
\newtheorem{example}[theorem]{Example}
\newtheorem{exercise}[theorem]{Exercise}
\newtheorem{condition}[theorem]{Condition}
\newtheorem{remark}[theorem]{Remark}
\newenvironment{proof}{\paragraph{Proof:}}{\hfill$\square$}

%Commands
\newcommand\independent{\protect\mathpalette{\protect\independenT}{\perp}}
\def\independenT#1#2{\mathrel{\rlap{$#1#2$}\mkern2mu{#1#2}}}
\newcommand{\overbar}[1]{\mkern 1.5mu\overline{\mkern-1.5mu#1\mkern-1.5mu}\mkern 1.5mu}
\newcommand{\equald}{\ensuremath{\overset{d}{=}}}

%Logo
%\AddToShipoutPictureBG{%
 % \AtPageUpperLeft{\raisebox{-\height}{\includegraphics[width=1.5cm]{uchicago.png}}}
%}




\begin{document}

\title{\textbf{Practical Computing for Economists}\footnote{This course does not count as a general distribution requirement and does not assign other grade than ``R''. }}
\author{Spring 2015}
\date{First Draft: July 24 , 2014 \\ This Draft: \today}
\maketitle

\begin{abstract}
\noindent The objective of this course is to provide students with the basic knowledge economists need when performing numerical economic analysis and estimating econometric models.  We focus on two programming languages: R, a mature language with extensive existing packages and debugging support, and Julia, the newest and fastest user-friendly numerical computing language. The course will provide economists and econometricians with the essential computational toolkit to conduct high-level research efficiently and accurately. Comprehensive example applications cover essential topics such as numerical integration, numerical likelihood maximization, and bootstrapping. \end{abstract}


\section{A Rapid Introduction to R for Economists}
\noindent \textbf{Instructor}: John Eric Humphries.\\
\noindent \textbf{Time}: Two weeks.\\
\noindent \textbf{Instruction Material}: Available on \href{http://johnerichumphries.com}{http://johnerichumphries.com}.\\
\noindent \textbf{Description:} \noindent \textbf{Description:} this section provides a rapid introduction to the R environment for statistical computing. The objective of this part of the course is two part: (1) equip studets with the skills to use R as a statistical computing environment and (2) equip students with the skills to program in R so that they can implement their own models and methods. Practical skills covered include understanding R code, writing well documented R code, visualizing and summarizing data, writing functions, control flow, and calling optimizers -- no prior experience programming in R or other languages is required for this portion of the course. 

\section{An Introduction to Rapid R for Economists}
\noindent \textbf{Instructor}: Philip Barrett.\\
\noindent \textbf{Time}: Two weeks.\\
\noindent \textbf{Instruction Material}: \\
\noindent \textbf{Description:} this section is about showing you how to speed up your R code.  We will very briefly discuss parallelism, but our focus will be on learning how to rewrite loops and linear algebra in C++.  These are often the most computationally intensive parts of a large project and can be pushed to C++ very easily and with substantial speed gains - improvements of 60 to 100 times faster are not uncommon.  We will use the \tt Rcpp \rm package to write C++ code that we can call from directly R, and the \tt RcppArmadillo \rm package to use very fast C++ linear algebra libraries with really easy syntax.


\section{Advanced Econometrics in Julia: Quickly Estimate Any Model}
\noindent \textbf{Instructor}: Bradley J. Setzler.\\
\noindent \textbf{Time}: Two weeks.\\
\noindent \textbf{Instruction Material}: Available on \href{JuliaEconomics.com}{JuliaEconomics.com}.\\
\noindent \textbf{Description}: This section introduces students to the Julia programming language. Created in 2012, Julia aims to be the fastest among the user-friendly numerical programming languages. I demonstrate how to copy your theoretical equations from \LaTeX\ into Julia to achieve high-speed estimation without having to change the logic of your equations (Julia even permits \LaTeX\ symbols). Julia is a function-friendly language (economics is a function-based field of study), and you can run nested loops without worrying about speed, following the standard notation of economics ($\forall i, \forall t$, etc.). I show how to run Julia in the cluster with a single line of code, that is, you can run your code on 150 CPU's of the University server simultaneously after only a one-line command. I use very computationally intensive estimation routines (nonlinear recursive Bayesian filters, high-dimensional fixed effects, bootstrap step-down multiple hypothesis testing) to show that Julia can rapidly compute (almost) any challenge we can imagine as economists.


\section{Advanced Econometrics in Julia: Applied Structural Models}
\noindent \textbf{Instructor}: Jorge L. Garc\'{i}a.\\
\noindent \textbf{Time}: Two weeks.\\
\noindent \textbf{Instruction Material}: \href{http://home.uchicago.edu/~jorgelgarcia/}{\begin{scriptsize}http://home.uchicago.edu/~jorgelgarcia\end{scriptsize}}.

\noindent \textbf{Description}: In this section structural models in Economics will be described briefly. In addition, structural and non-structural estimation will be recognized as approaches that enable the user to answer different research questions. Also, parametric and non-parametric assumptions will be discussed as auxiliary means that contribute to recover parameters. This section of the course will build on \citet{keane2011structural} but will be expanded by providing a wide range of computational implementations of their baseline model – a classic discrete choice dynamic programming model of women's labor supply –. The program Julia will be used for these implementations and additional some Python spinets will be provided for the sake of comparison.


\bibliographystyle{chicago}
\bibliography{BibtexFiles/Syllabus}
\clearpage

\end{document}
